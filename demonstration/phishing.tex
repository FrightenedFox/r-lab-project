\documentclass{beamer}
\usepackage{pgf} % - pozwala wczytywać formaty obrazków jak np. .jpg
%\usepackage{polski} % - polskie znaki diakrytyczne
\usepackage[utf8]{inputenc}
\usepackage{xcolor} % - dodatkowa gama kolorów
\usetheme{Goettingen} % - styl prezentacji
\usecolortheme{seahorse} % - kolorystyka prezentacji
\usefonttheme{default} % - styl czcionki w prezentacji
\useoutertheme{tree} % - sty=l nagłówka i stopki

\setbeamercolor{alerted text}{fg=red}
\setbeamercolor{background canvas}{bg=white}
\setbeamercolor{block body alerted}{bg=red!90!black}
\setbeamercolor{block body}{bg=blue!20!white}
\setbeamercolor{block body example}{bg=normal text.bg!90!black}
\setbeamercolor{block title alerted}{use={normal text, alerted text},
	fg=alerted text.fg!15!black, bg=alerted text.fg!75!black}
\setbeamercolor{block title}{bg=gray!30}
\setbeamercolor{block title example}{use={normal text, example text},
	fg=example text.fg!15!black, bg=example text.fg!75!black}
\setbeamercolor{frametitle}{fg=black}
\setbeamercolor{item projected}{fg=black}
\setbeamercolor{normal text}{bg=black}

\setbeamertemplate{theorems}[numbered]
\newtheorem{tw}{Twierdzenie}
\newtheorem{lm}{Lemat}
\newtheorem{uw}{Uwaga}
\newtheorem{pr}{Przykład}
\newtheorem{wn}{Wniosek}
\theoremstyle{definition}
\newtheorem{df}{Definicja}
\def\proofname{Dowód}

\usepackage{pgf,tikz,pgfplots}
\pgfplotsset{compat=1.15}
\usepackage{mathrsfs}
\usetikzlibrary{arrows}

\usepackage{amssymb}
\usepackage{graphicx}
\usepackage{geometry}
\usepackage{amsmath}
\usepackage{geometry}
\usepackage{latexsym}
\usepackage{wrapfig}
\usepackage{multicol}
\usepackage{qtimes}
\usepackage{pgfplots}
\usepackage{xcolor}
\usepackage{qtimes}

\title{Subdivision of graphs in $\mathcal{R}(mK_2,P_4)$}
\author {Julia Makarska}
\date[\empty]

\begin{document}
	
\begin{frame}
	\maketitle
\end{frame}
\begin{frame}[shrink]
\frametitle{Table of contents}
	\hypertarget{a}\tableofcontents
\end{frame}
\section{Abstract}
\begin{frame}
	\uncover<1->
		{
		In the following paper, we will discuss how to create new Ramsey minimal graphs from previous known Ramsey minimum  graphs using the subdivision operation.\\~\\
		}
	\uncover<2->
	{
		Let us denote $F$, $G$, $H$ as graphs. The notation $F\to(G,H)$ means that red-blue coloring of all edges of $F$ will contain either a red copy of $G$ or a blue copy of $H$.\\~\\
	}
	\uncover<3->
	{
		The set $\mathcal{R}(G,H)$ consists of all Ramsey $(G,H)$ -- minimal graphs.
	}
\end{frame}

\section{Introduction}
	\subsection{Lemma 1}
\begin{frame}
	\uncover<1->
	{
		\begin{block}{Lemma 1}
		Let $H$ be a connected graph and $m$ be a positive integer. 
		Suppose $F\in\mathcal{R}(mK_2,H)$. For each $e\in E(F)$, let $\tau$ be an $(mK_2,H)$-coloring of edges of $F-e$. Then, there exists a red $(m-1)K_2$ in $F-e$.
		\end{block}
	}
\end{frame}

\section{Subdivision graphs}
	\subsection{Example - subdivision of the graph $G$}
\begin{frame}
	\uncover<1->
	{
		\begin{block}{Example 1}
		Consider a graph $G$ shown in the Figure \ref{fig:f1}. Next to the graph $G$ can be seen subdivision of the graph $G$ on the edge $e_4$ and $e_6$.
		\end{block}
	}
	\uncover<2->
	{
	\begin{figure}[h]
		\centering
		\includegraphics[width=0.7\linewidth]{Figure_1}
		\caption{The graphs $G$, $SG(e_6,2)$, $SG(e_4,2)$}
		\label{fig:f1}
	\end{figure}
}
\end{frame}
\subsection{Lemma 2}
\begin{frame}
	\uncover<1->
		{
			\begin{block}{Lemma 2}
			Let $m\geq 2$ be an integer and $F\in\mathcal{R}(mK_2,P_4)$. Then, for any $e\in E(F)$, there exists a red-blue coloring of $F$ having no red $mK_2$ and the edge $e$ satisfies one of the following four conditions:
			\begin{enumerate}
				\item[i)] $e$ is any edge of exactly one blue path $P_4$,
				\item[ii)] $e$ is the middle edge of more than one blue path $P_4$ (there is no blue path $P_5$ in this case)
				\item[iii)] $e$ is one of the middle edges of one or more than one blue path $P_5$ (there is no blue path $P_6$ in this case), or
				\item[iv)] $e$ is the middle edge of one or more than one blue path $P_6$. \textup{\cite{1}}
			\end{enumerate}
			\end{block}
		}
\end{frame}
\begin{frame}
		\begin{figure}[h]
			\centering
			\includegraphics[width=0.7\linewidth]{Figure_2}
			\caption{The four conditions}
			\label{fig:f2}
		\end{figure}
\end{frame}
	\subsection{Theorem 1}
\begin{frame}
	\uncover<1->
	{
		\begin{block}{Theorem 1}
			Let $F$ be a connected graph and $m\geq2$ be an integer. Suppose $\alpha$ is an edge contained in a cycle of $F$. If $F\in\mathcal{R}(mK_2,P_4)$, then $SF(\alpha,4)\in\mathcal{R}(E(F))$. Consequently, $SF(4)\subseteq\mathcal{R}\left( (m+1)K_2,P_4 \right).$ \textup{\cite{1}}
		\end{block}
	}
\end{frame}
\section{Conclusions}
\begin{frame}
	In conclusion, we have covered just a few of the basics of a given topic. The driving itself is difficult but very interesting. Therefore, I encourage you to review the bibliography given below for yourself.
\end{frame}
		

\section{References}
\begin{frame}[allowframebreaks]{Literatura}
\begin{thebibliography}{}
	\setbeamertemplate{bibliography item}[online]
	\bibitem{1} 
	Kristiana Wijaya, Edy Tri Baskoro, Hilda Assiyatun, Djoko Suprijanto, \textit{Subdivision of graphs in $\mathcal{R}(mK_2,P_4)$}
\end{thebibliography}
\end{frame}

\end{document}